\documentclass{article}
\usepackage{url}
\title{Manuale d'uso}
\author{Francesco Paolo Castiglione}
\date{2020-2021}
\renewcommand{\contentsname}{Indice}
\begin{document}
\maketitle

\tableofcontents

\section{Avvio dell'applicazione}

Per avviare la web application seguire la procedura qui riporatata:

\begin{itemize}
	\item Impostare ed avviare il database \texttt{MySQL}
	\begin{enumerate}
		\item Eseguire lo script \texttt{SQL\_DB\_CREATION.sql} per creare e popolare il database
		\item Modificare i campi \texttt{username}, \texttt{password} e \texttt{url} nella sezione \texttt{Resource} ed il campo \texttt{connectionURL} nella sezione \texttt{Realm} nel file \texttt{"/META-INF/context.xml"} affinchè coincidano con i dati della macchina sulla quale si sta avviando la web application.\newline	
		I valori di default sono:
		\begin{itemize}
			\item Database: lido\_test
			\item Username: root
			\item Password: password
		\end{itemize}
		\item Avviare il server \texttt{MySQL}
	\end{enumerate}
	\item Impostare ed avviare \texttt{Apache Tomcat}
	\begin{enumerate}
		\item La web application è stata sviluppata e testata su \texttt{Apache Tomcat} versione \texttt{9.0.38}. Versioni differenti potrebbero portare ad eventuali problemi di compatibilità nel formato di \texttt{"/META-INF/context.xml"} e \texttt{"/META-INF/web.xml"}
		\item Posizionare \texttt{Lido.war} all'interno della cartella \texttt{webapps} di \texttt{Apache Tomcat}
		\item Spostare il file \texttt{.jar} \texttt{"/WEB-INF/lib/my-sql-connector-java-8.0.13.jar"} nella cartella \texttt{"\$CATALINA\_HOME/lib"}
		\item Avviare il servizio di \texttt{Apache Tomcat}
		\item Visitare l'url \url{http://localhost:8080/Lido/}
	\end{enumerate}
\end{itemize}

\section{Creazione account}


\end{document}